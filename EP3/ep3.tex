\documentclass[leqno]{article}
\usepackage[a4paper, margin=1.3in]{geometry}
\usepackage[utf8]{inputenc}
\usepackage{amsmath}
\usepackage{graphicx}
\usepackage{caption}
\usepackage{subcaption}
\usepackage{float}
\usepackage{indentfirst}

\title{MAC0210 - Exercício-Programa 3}
\author{Vinicius Agostini - 4367487}
\date{Julho 2020}

\begin{document}

\maketitle

\section{Computando Trabalho}
A primeira tarefa do EP é aproximar a quantidade de trabalho exercida estimando
a integral $\int_{x_0}^{x_n}F(x)\cos(\theta(x))dx$, primeiro interpolando os valores
fornecidos. O método de interpolação escolhido foi o método de Lagrange. Então,
devemos usar esses valores para estimar a integral usando a regra do
trapézio composto e a regra de Simpson composto.

Para fazer a interpolação, decidi dividir o intervalo $[0,30]$ em $n-1$ intervalos
e calculei os $n$ pontos resultantes através de uma função que executa a interpolação
pelo método de Lagrange. \\

Resultados:
\begin{table}[H]
    \centering
    \begin{tabular}{c|c|c|c|c|c|}
    \cline{2-6}
                                      & $n = 7$    & $n = 19$   & $n = 61$   & $n = 100$  & $n = 500$  \\ \hline
    \multicolumn{1}{|c|}{Trapezoidal} & 119.089250 & 117.347276 & 117.151007 & 117.138741 & 117.131902 \\ \hline
    \multicolumn{1}{|c|}{Simpson}     & 117.127167 & 117.069079 & 117.131612 & 117.069079 & 117.123476 \\ \hline
    \end{tabular}
    \end{table}

Inicialmente achei que deveria utilizar valores de $n$ tal que $n = 6k + 1$ para
obter melhores aproximações já que dessa forma garantiríamos que os pontos originais
estariam presentes no conjunto de pontos interpolados e que isso seria uma coisa
boa, porém, conforme $n$ aumenta, isso não se mostrou um fator tão importante.
No caso da regra de Simpson, o erro relacionado a utilizar uma quantidade par
ou ímpar de intervalos também se torna negligenciável conforme $n$ aumenta.
\\

\textbf{Uso do programa:}

\texttt{gcc trabalho.c -o trabalho}

\texttt{./trabalho a b n}, para calcular $\int_a^b F(x)\cos(\theta(x))dx$ com
$n$ pontos, interpolando a partir da tabela do enunciado.

\textbf{Exemplo:}

\texttt{./trabalho 0 30 19} (1 ponto interpolado entre cada ponto dado).

\section{Integração por Monte Carlo}

\subsection{Integrais unidimensionais}
Nesta seção do EP devemos calcular uma aproximação para integrais unidimensionais
como $I = \int_0^1 g(x) dx$ escolhendo pontos aleatórios $U_i$ no intervalo $[0,1]$ e
calculando $\hat{I} = \sum_{i=1}^n \frac{g(U_i)}{n}$, que, pela Lei Forte dos
Grandes Números, se aproxima de $I$ quando $n \to \infty$. No caso de integrais
em intervalos arbitrários $[a,b]$ é possível realizar uma troca de variável para
resolver uma integral no intervalo $[0,1]$ que tenha resultado equivalente. \\

Integrais a serem resolvidas e trocas de variável utilizadas:
\begin{align}
    &\int_0^1 \sin(x) dx \\
    \int_3^7 x^3 dx &= \int_0^1 4 \cdot (4y+3)^3 dy, \qquad \text{com } y = \frac{x-3}{4}\\
    \int_0^\infty e^{-x} dx &= \int_0^1 \frac{e^{1-\frac 1y}}{y^2} dy, \qquad \text{com } y = \frac{1}{1+x}
\end{align}


Resultados:
\begin{table}[H]
    \centering
    \begin{tabular}{c|c|c|c|c|c|}
    \cline{2-6}
                            & $10^4$ & $10^5$ & $10^6$ & $10^7$ & Valor esperado \\ \hline
    \multicolumn{1}{|c|}{$\int_0^1 \sin(x) dx$} &  0.456837          &  0.459729          &  0.459457      &  0.459850   & $\approx$ 0.45970  \\ \hline
    \multicolumn{1}{|c|}{$\int_0^1 4 \cdot (4y+3)^3 dy$} &  575.820989        &  580.472969        &  580.176466    &  580.232532 &  580               \\ \hline
    \multicolumn{1}{|c|}{$\int_0^1 \frac{e^{1-\frac 1y}}{y^2} dy$} &  0.996641          &  0.999422          &  0.999720      &  1.000108   &  1                 \\ \hline
    \end{tabular}
\end{table}

Para esta parte foi implementada uma função única para o método de Monte Carlo
que recebe uma função como parâmetro com assinatura
\textbf{monte\_carlo(int iters, function func)} onde \textbf{iters}
é a quantidade de iterações a serem realizadas e \textbf{func} uma função que
recebe um double e retorna um double. A definição do tipo \textbf{function} foi criada
a partir do comando \textbf{typedef double (*function)(double)}.
\\

\textbf{Uso do programa:}

\texttt{gcc monte\_carlo\_1d.c -o monte\_carlo\_1d}

\texttt{./monte\_carlo\_1d iters}

\textbf{Exemplo:}

\texttt{./monte\_carlo\_1d 1000}


\subsection{Integrais multidimensionais}
Aqui o objetivo é encontrar uma estimativa para o valor de $\pi$. O método utilizado
é considerar a parte do círculo unitario que está no primeiro quadrante e calcular
sua área por Monte Carlo da seguinte forma, como definido no enunciado:

\begin{align*}
    g(x,y) &= \begin{cases}
        1, \text{   se } x^2 + y^2 \leq 1 \\
        0, \text{   caso contrário}
    \end{cases}\\
    \\
    \pi &= 4 \cdot \int_0^1\int_0^1 g(x,y)dxdy
\end{align*}

Resultados:
\begin{table}[H]
    \centering
    \begin{tabular}{c|c|}
    \cline{2-2}
                                     & $\approx \pi$ \\ \hline
    \multicolumn{1}{|c|}{$n = 10^4$} & 3.142400      \\ \hline
    \multicolumn{1}{|c|}{$n = 10^5$} & 3.144680      \\ \hline
    \multicolumn{1}{|c|}{$n = 10^6$} & 3.140356      \\ \hline
    \multicolumn{1}{|c|}{$n = 10^7$} & 3.141540      \\ \hline
    \end{tabular}
\end{table}

\textbf{Uso do programa:}

\texttt{gcc pi.c -o pi}

\texttt{./pi iters}

\textbf{Exemplo:}

\texttt{./pi 1000}

\end{document}