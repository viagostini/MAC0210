\documentclass[leqno]{article}
\usepackage[a4paper, margin=1.5in]{geometry}
\usepackage[utf8]{inputenc}
\usepackage{amsmath}

\title{MAC0210 - Capítulo 11}
\author{Vinicius Agostini - 4367487}
\date{May 2020}

\begin{document}

\maketitle

\section*{Exercício 5}

Dado $t_0 < t_1 < \ldots < t_n$, definimos uma \textit{spline cúbica} por:

$$ S(x) = S_i(x),\quad t_i \leq x \leq t_{i+1} $$

onde


$$ S_i(x) = a_ix^3 + b_ix^2 + c_ix + d_i,\quad i = 0, 1, \ldots, n-1. $$

Isso nos dá um total de $4n$ incógnitas que devemos descobrir para então definir
nossa $S(x)$.

Como $S(x)$ deve interpolar os pontos $t_i$, sabemos que devemos ter
$S_i(x) = y_i$ e, além disso, uma \textit{spline cúbica} deve ser duplamente
diferenciável, resultando nas seguintes equações:

\begin{align}
    &S_i(t_i) = y_i                       &i = 0, 1, \ldots, n-1\\
    &S_i(t_{i+1}) = y_{i+1}               &i = 0, 1, \ldots, n-1\\
    &S_i'(t_{i+1}) = S_{i+1}'(t_{i+1})    &i = 0, 1, \ldots, n-2\\
    &S_i''(t_{i+1}) = S_{i+1}''(t_{i+1})  &i = 0, 1, \ldots, n-2
\end{align}

Essas condições nos dão um total de $4n-2$ equações que nos garantem que $S(x)$
é interpoladora e que $S'(x)$ e $S''(x)$ são contínuas. Porém, são necessárias
mais duas equações para que possamos resolver o sistema e encontrar os
coeficientes para $S(x)$.

Aqui, as duas últimas equações são uma escolha livre e, para este exercício,
vamos escolher as condições de \textit{not-a-knot}, ou seja:

\begin{align}
    &S_0'''(t_1) = S_1'''(t_1)\\
    &S_{n-2}'''(t_{n-1}) = S_{n-1}'''(t_{n-1})
\end{align}

\subsubsection*{Construção de $S(t)$:}
Definindo $z_i = S''(t_i)$, temos pelas condições acima que, nos pontos interiores:
        $$ S_i''(t_i) = S_{i+1}''(t_i) = z_i \qquad i = 1, \ldots, n-1. $$

como $S_i(x)$ é um polinômio cúbico, $S_i''(x)$ é uma função linear tal que
podemos escrever, na forma de Lagrange:

$$ S_i''(x) = \frac{z_{i+1}}{h_i}(x - t_i) - \frac{z_i}{h_i}(x - t_{i+1}) $$

onde $h_i = t_{i+1} - t_i$. Então podemos integrar para obtermos:

\begin{align*}
S_i'(x) &= \frac{z_{i+1}}{2h_i}(x-t_i)^2 - \frac{z_i}{2h_i}(x-t_{i+1})^2 + C_i + D_i\\
S_i(x)  &= \frac{z_{i+1}}{6h_i}(x-t_i)^3 - \frac{z_i}{6h_i}(x-t_{i+1})^3 + C_i(x-t_i) + D_i(x-t_{i+1}).
\end{align*}


Pela propriedade de interpolação $S_i(t_i) = y_i$, podemos então escrever:

$$ y_i = -\frac{z_i}{6h_i}(-h_i)^3 - D_i(-h_i) = \frac{1}{6}z_ih_i^2 + D_ih_i \implies D_i = \frac{y_i}{h_i} - \frac{h_i}{6}z_i.$$

Pela propriedade de interpolação $S_i(t_{i+1}) = y_{i+1}$, temos:

$$ y_{i+1} = \frac{z_{i+1}}{6h_i}h_i^3 + C_ih_i \implies  C_i = \frac{y_{i+1}}{h_i} - \frac{h_i}{6}z_{i+1}. $$

Logo, assim que pudermos determinar todos os $z_i$ teremos todos $S_i$ e $S_i'$
já que podemos escrevê-los como:

\begin{align*}
    \begin{split}
    S_i(x) &= \frac{z_{i+1}}{6h_i}(x-t_i)^3 - \frac{z_i}{6h_i}(x-t_{i+1})^3\\
    & \hspace{105pt} + \left( \frac{y_{i+1}}{h_i} - \frac{h_i}{6}z_{i+1} \right)(x-t_i)-\left( \frac{y_i}{h_i} - \frac{h_i}{6}z_i \right)(x-t_{i+1}).
    \end{split}\\
    S_i'(x) &= \frac{z_{i+1}}{2h_i}(x-t_i)^2 - \frac{z_i}{2h_i}(x-t_{i+1})^2 + \frac{y_{i+1} - y_i}{h_i} - \frac{z_{i+1} - z_i}{6}h_i.
\end{align*}

As condições de continuidade de $S'(x)$ necessitam que:

\begin{align*}
    S_{i-1}'(t_i) &= S_i'(t_i)           &i = 1,2,\ldots, n-1\\
    S_i'(t_i)     &= -\frac{z_i}{2h_i}(-h_i)^2 + \underbrace{\frac{y_{i+1} - y_i}{h_i}}_{b_i} - \frac{z_{i+1} - z_i}{6}h_i\\
                  &= -\frac 16 h_iz_{i+1} - \frac 13 h_iz_i + b_i\\
    S_{i-1}'(t_i) &= \frac 16 z_{i-1}h_{i-1} + \frac 13 z_ih_{i-1} + b_{i-1}\\
    \intertext{Igualando as equações e manipulando algebricamente:}
    h_{i-1}z_{i-1} + 2(h_{i-1} &+ h_i)z_i + h_iz_{i+1} = 6(b_i - b_{i-1})     &i=1,2,\ldots,n-1
\end{align*}

Pelas condições not-a-knot temos que $S_0 \equiv S_1$ e $S_{n-2} \equiv S_{n-1}$
já que são iguais até a terceira derivada.

Então temos o sistema:

$$ \begin{cases}
    h_{i-1}z_{i-1} + 2(h_{i-1} + h_i)z_i + h_iz_{i+1} = 6(b_i - b_{i-1})   \qquad \qquad  i=1,2,\ldots,n-2\\
    z_0 = z_1\\
    z_{n-2} = z_{n-1}
\end{cases} $$

que vamos escrever na forma $H \cdot \vec{z} = \vec{b}$:

$$
\begin{pmatrix}
    3h_0 + 2h_1 & h_1          &                             \\
    h_1         & 2(h_1 + h_2) & h_2                         \\
%                & h_2          & 2(h_2 + h_3)   & h_3        \\
                & \ddots       & \ddots         & \ddots     \\
                &              & h_{n-4}        & 2(h_{n-4} + h_{n-3}) & h_{n-3} \\
                &              &                & h_{n-3}        & 2(h_{n-3} + h_{n-2}) & h_{n-2}
\end{pmatrix}
\cdot
\begin{pmatrix}
    z_1\\
    z_2\\
%    z_3\\
    \vdots\\
    z_{n-3}\\
    z_{n-2}
\end{pmatrix}
=
\begin{pmatrix}
    6(b_1-b_0)\\
    6(b_2-b_1)\\
%    6(b_3-b_2)\\
    \vdots\\
    6(b_{n-3}-b_{n-4})\\
    6(b_{n-2}-b_{n-3})
\end{pmatrix}
$$

Podemos ver que a matriz $H$ é tridiagonal e também estritamente diagonal dominante
já que 
$
\begin{cases}
    3h_0 + 2h_1 > |h_0 + h_1| > |h_1|\\
    2 |h_{i-1} + h_i| > |h_{i-1}| + |h_i|
\end{cases}
$.

Uma vez que este sistema for resolvido, obtemos $z_0$ e $z_{n-1}$ a partir das
condições not-a-knot que estabelecemos anteriormente.
%Utilizando as igualdades \textit{not-a-knot} podemos escrever $S_i'''(x)$ para
%s casos em que garantimos a sua existência, então:

%$$ S_1'''(t_0) = \frac{z_2}{h_1} - \frac{z_1}{h_1} $$

\end{document}
